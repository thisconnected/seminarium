% Created 2021-03-15 pon 18:40
% Intended LaTeX compiler: pdflatex
\documentclass[11pt]{article}
\usepackage[utf8]{inputenc}
\usepackage[T1]{fontenc}
\usepackage{graphicx}
\usepackage{grffile}
\usepackage{longtable}
\usepackage{wrapfig}
\usepackage{rotating}
\usepackage[normalem]{ulem}
\usepackage{amsmath}
\usepackage{textcomp}
\usepackage{amssymb}
\usepackage{capt-of}
\usepackage{hyperref}
\author{Patryk Kaniewski}
\date{\today}
\title{Emulacja 8086 używając nowoczesnego jezyka C++}
\hypersetup{
 pdfauthor={Patryk Kaniewski},
 pdftitle={Emulacja 8086 używając nowoczesnego jezyka C++},
 pdfkeywords={},
 pdfsubject={},
 pdfcreator={Emacs 27.1 (Org mode 9.3)}, 
 pdflang={English}}
\begin{document}

\maketitle

\section{Cel}
\label{sec:org42f9c59}
Zbudowanie biblioteki do emulacji 8086 w C++17, a nastepnie zbudowanie na jej podstawie emulatora 8086 ktory mozna byłoby wykonywać proste programy DOS.
\section{Motywacja}
\label{sec:orgf9c36b1}
8086 jest jednoczesnie dostyć prostą architektura (ograniczona liczba instrukcji w porównaniu do nowoczesnego x86 z dużą liczbą rozszerzeń) i posiada tylko realmode
\section{Używane technologie}
\label{sec:org67100b1}
\begin{itemize}
\item C++17
\item GCC
\item cmake
\item tinyasm
\item ncurses
\end{itemize}
\section{Spis czynnosci}
\label{sec:org80575b2}
\begin{itemize}
\item disasembler 8086
\item zbudowanie toolchaina (uzywajac nasm i bash) do latwego assemblowania i uruchamiania kodu 8086
\item zbudowanie systemu debugowania (register dump, memory dump, single step)
\item interpretacja instrukcji 8086
\item implementacja wywolan systemowych DOS/BIOS
\item implementacja instrukcji zmiennoprzecinkowych 8087
\item wolf3d
\end{itemize}
\section{Spis tresci}
\label{sec:org16d53d7}
\begin{enumerate}
\item Wprowadzenie
\item Budowa środowiska
\begin{enumerate}
\item Disasembler
\item Toolchain i asembler
\item Debugger
\end{enumerate}
\item Implementacja emulatora
\item Urządzenia peryferyjne
\begin{enumerate}
\item Video mode
\item Klawiatura
\item Dyskietka
\end{enumerate}
\item Wyniki
\item Podsumowanie
\end{enumerate}
\section{Referencje}
\label{sec:org6719200}
\subsection{SoK: All You Ever Wanted to Know About x86/x64 Binary Disassembly But Were Afraid to Ask}
\label{sec:org879cf05}
\url{https://arxiv.org/abs/2007.14266}
\subsection{Verifying x86 Instruction Implementations}
\label{sec:org976882f}
\url{https://arxiv.org/abs/1912.10285}
\subsection{The INTEL® 8087 numeric data processor}
\label{sec:org00517cf}
\url{https://dl.acm.org/doi/10.1145/1500518.1500674}
\subsection{Design and Implementation Techniques of the 8086 C Decompiling System}
\label{sec:orgafaa8b4}
\url{https://apps.dtic.mil/sti/citations/ADA294633}
\subsection{Formal Specification of the x86 Instruction Set Architecture}
\label{sec:orgf1e4221}
\url{https://publikationen.sulb.uni-saarland.de/handle/20.500.11880/26394}
\end{document}
